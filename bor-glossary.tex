\newglossaryentry{due-process}{name={due process of law},description={\foreignlanguage{swedish}{vederbörlig rättegång, rättssäkerhet}; \foreignlanguage{finnish}{asianmukainen oikeudenkäynti}; \foreignlanguage{french}{le procès dû de la loi}; \foreignlanguage{spanish}{el proceso debido de la ley;}} everything in order as it ought to be according to the procedures established by law}
\newglossaryentry{twice-jeopardy}{name={twice placed in jeopardy},description={see \gls{double-jeopardy}}}
\newglossaryentry{double-jeopardy}{name={double jeopardy},description={a situation of facing criminal charges for the second time for the same incident; ``jeopardy'' derived from French \foreignlanguage{french}{\textit{jeu-parti}}}}
\newglossaryentry{congress}{name={congress},description={samlag; yhdistys, se on, Yhdys-Valtojen korkein lainsäädäntäyhdistys; le congrès; when capitalized, the proper name of the United States' bicameral law-making body}}
\newglossaryentry{poll-tax}{name={poll tax},description={\foreignlanguage{swedish}{skatt per capita}; \foreignlanguage{finnish}{henkilömäärävero}; \foreignlanguage{french}{capitation}; a direct tax levied per capita; i.e., at a fixed dollar amount per person without regard to assets or income}}
\newglossaryentry{legislature}{name={legislature},description={\foreignlanguage{swedish}{lagstiftande församling}; \foreignlanguage{finnish}{lainsäädäntäelin}; a law-making body}}
\newglossaryentry{infamous-crime}{name={capital, or otherwise infamous crime},description={a serious crime; any crime punishable by death or otherwise having the characteristic of \textit{infamy}. All felonies, and certain other crimes, are infamous}}
\newglossaryentry{presentment}{name={presentment},description={\foreignlanguage{swedish}{föreläggande}; \foreignlanguage{finnish}{löydettäminen}; \foreignlanguage{french}{présentation}; a finding, or the presentation of a finding, of a jury}}
\newglossaryentry{indictment}{name={indictment},description={\foreignlanguage{swedish}{åtal}; ; \foreignlanguage{french}{mise en cause}; a formal charge for a serious crime}}
\newglossaryentry{indict}{name={indict},description={\foreignlanguage{swedish}{åtala}}; see \gls{indictment}}
\newglossaryentry{grand-jury}{name={grand jury},description={a jury convened to examine the evidence of serious criminal charges, to determine whether or not the charges on a bill of indictment are brought in good faith on probable cause, and if so to return a finding of ``a true bill,'' if not, ``not a true bill;'' as distinguished from a \gls{petit-jury}}}
\newglossaryentry{petit-jury}{name={petit jury},description={a jury that \glsdisp{try}{tries} the particulars of a criminal matter; a felony or other infamous crime, except in active military service, must be \glsdisp{indict}{indicted} by a \gls{grand-jury} before \gls{trial}}}
\newglossaryentry{actual-service}{name={when in actual service},description={\foreignlanguage{swedish}{när i pågående tjänst}; \foreignlanguage{finnish}{ollen nykyaikaisessa palveluksessa}; \foreignlanguage{french}{quant à service actuelle}; \textit{relevant} and \textit{timely} to military service, that is, as to or relating to ongoing service at the actual time; the case of a summarily convicted ``traitor'' or ``coward'' or one either refusing to fight or actively fighting on the side of the enemy in battle, whose punishment is that inflicted not of criminal justice but of necessity on an enemy combatant in war}}
\newglossaryentry{oath}{name={oath},description={a sworn statement}}
\newglossaryentry{affirmation}{name={affirmation},description={alternative to \gls{oath}, for those with a religious or philosophical objection to swearing, e.g., James~5:12}}
\newglossaryentry{abridge}{name={abridge},description={\foreignlanguage{swedish}{förkorta}; \foreignlanguage{finnish}{lyhentää}; \foreignlanguage{french}{abréger}; to shorten, curtail, constrain, limit, or restrict}}
\newglossaryentry{privilege}{name={privilege},description={an internationally recognized legal term}}
\newglossaryentry{immunity}{name={immunity},plural={immunities},description={a diplomatic legal term}}
\newglossaryentry{circumcise}{name={circumcise},description={\foreignlanguage{swedish}{omskara}; \foreignlanguage{finnish}{ympärileikata}; \foreignlanguage{french}{circoncire}; to ``cut around,'' i.e., to amputate the foreskin and cauterize the sebaceous glands of the glans penis, as required in the U.S., or otherwise mutilate or cut off some or all of the external genitalia of either sex}}
\newglossaryentry{ratify}{name={ratify},description={\foreignlanguage{swedish}{ratificera}; to formally approve a treaty; said of a sovereign state}}
\newglossaryentry{try}{name={try},description={\foreignlanguage{swedish}{pröva}; \foreignlanguage{finnish}{koettaa}; \foreignlanguage{french}{éprouver}; see \gls{trial}}}
\newglossaryentry{trial}{name={trial},description={\foreignlanguage{swedish}{prövning}; \foreignlanguage{finnish}{koetus}; \foreignlanguage{french}{épreuve}; debate between prosecution and defense in a civil or criminal matter before a jury, which hears the evidence, deliberates, and returns a verdict}}
\newglossaryentry{senator}{name={senator},description={\foreignlanguage{swedish}{senator}; \foreignlanguage{finnish}{yliherra}; \foreignlanguage{french}{sénateur}; when capitalized, the title of a member of the \Gls{senate}}}
\newglossaryentry{representative}{name={representative},description={\foreignlanguage{swedish}{representat}; \foreignlanguage{finnish}{edustaja}; \foreignlanguage{french}{représentative}; when capitalized, the title of a member of the \glsdisp{house-of-reps}{House of Representatives}}}
\newglossaryentry{house-of-reps}{name={house of representatives},description={\foreignlanguage{swedish}{representathuset}; \foreignlanguage{finnish}{edustajainhuone}; \foreignlanguage{french}{la chambre des représentatives}; when capitalized, the proper name of the lower house of \Gls{congress}}}
\newglossaryentry{senate}{name={senate},description={\foreignlanguage{swedish}{senat}; \foreignlanguage{finnish}{yhdistyksen ylähuone}; \foreignlanguage{french}{le sénat}; when capitalized, the proper name of the upper house of \Gls{congress}}}
\newglossaryentry{keep}{name={keep},description={\foreignlanguage{swedish}{äga, innehava}; \foreignlanguage{finnish}{omistaa, pitää}; \foreignlanguage{french}{tenir, posséder}; in modern U.~S.{} federal administrative law, to possess}}
\newglossaryentry{bear}{name={bear},description={\foreignlanguage{swedish}{bära}; \foreignlanguage{finnish}{kantaa}; \foreignlanguage{french}{porter}; in modern U.~S.{} federal administrative law, to carry}}
\newglossaryentry{arm}{name={arm},description={\foreignlanguage{swedish}{vapen, skjutvapen, eldvapen}; \foreignlanguage{finnish}{ase, ampuma-ase, tuli-ase}; \foreignlanguage{french}{arme, arme de jet, arme à feu}; a weapon, especially a projectile weapon or firearm}}
\newglossaryentry{infringe}{name={infringe},description={\foreignlanguage{swedish}{inskränka, överträda}; \foreignlanguage{finnish}{loukata}; \foreignlanguage{french}{enfreindre}; to trespass, encroach or make inroads (\foreignlanguage{swedish}{inskränkningar}) upon, or interfere with}}
\newglossaryentry{militia}{name={militia},description={\foreignlanguage{swedish}{milis}; \foreignlanguage{finnish}{miliisi}; \foreignlanguage{french}{milice}; an abstract (rather than concrete) body of armed men capable of defending the country, with a certain potentiality to its existence which does not come into actual effect until ``the men'' are called or (in older language) mustered or (in newer language) inducted; (almost all too neatly) administered by the Selective Service System in modern U.~S.{} federal law --- in a whorish agenda to ``register'' all men in general, all firearms, and as many men as possible as \textit{sex offenders}}}
\newglossaryentry{necessary}{name={necessary},description={\foreignlanguage{swedish}{nödig}; \foreignlanguage{finnish}{tarpeellinen}; \foreignlanguage{french}{nécessaire}; of a necessity, needful; the necessity (of a militia) being that of the people, not of the security apparatus of the state}}
\newglossaryentry{witness}{name={witness},plural={witnesses},description={\color{red}\foreignlanguage{swedish}{vittne}; \foreignlanguage{finnish}{vierasmies, todistaja}; \foreignlanguage{french}{témoin}; a noun: either the action \foreignlanguage{swedish}{<<\,vittnesbörd\,>>} or in this document rather the person \foreignlanguage{swedish}{<<\,vittne\,>>} performing the action of the English verb \textit{to wit}; (\foreignlanguage{swedish}{veta};  \foreignlanguage{finnish}{tietää}; \foreignlanguage{french}{savoir};) rather than \textit{to know}; (\foreignlanguage{swedish}{känna};  \foreignlanguage{finnish}{tuntea}; \foreignlanguage{french}{connaître}.) The verb \textit{to wit} is archaic and conjugated irregularly. In the infinitive it is always immediately preceded by the particle \textit{to}, and never negated. Its present participle, \textit{witting}, is never used as a gerund, but often appears in adverbial form \textit{wittingly}, and these are negated by prepending either the word \textit{not} or the prefix \mbox{\textit{un-}}, or both to make a double negative with a positive meaning.  There are no other tenses or moods for the verb ``to wit'' other than the simple present and simple past.  No auxiliary verbs are used with this verb and no adverb modifies it, except for the simple negative \textit{not}, placed immediately after the verb, or after the subject, if the subject is a personal pronoun and itself immediately follows the verb in a question or other inversion. In the simple present: I \textit{wot}, thou \textit{wost} (or \textit{wottest}), he/she \textit{wot} (or \textit{wotteth}), we/ye/they \textit{wit}.  In the simple past, I/thou/he/she/we/ye/they \textit{wist}.  The Finnish term \foreignlanguage{finnish}{\textit{vierasmies}} was already archaic by the end of the 19th century. It is a compound of the adjective \foreignlanguage{finnish}{\textit{vieras}}, \textit{strange}, and \foreignlanguage{finnish}{\textit{mies}}, \textit{man}, so ``strangeman'' --- by implication, one who is impartial and about whose testimony the jury is to be impartial as well.  Both parts are declined, \foreignlanguage{finnish}{\textit{vieraanmiehen}}, \foreignlanguage{finnish}{vieraillemiehille}, etc.  The other possible literal meaning of this word is ``guestman'' --- someone whom the court has the obligation of a host to put up in a safe house or offer reasonable protection in case of need.  In this sense a male witness' wife would be called by the feminine form of this word, \foreignlanguage{finnish}{\textit{vierasvaimo}}, but the husband of a female witness would have to be called \foreignlanguage{finnish}{\textit{vieraanmiehen mies}}, although there would be a severe implication that her testimony must not be unduly influenced by her husband.  The modern Finnish term \foreignlanguage{finnish}{\textit{todistaja}}, also has a possible feminine form, \foreignlanguage{finnish}{\textit{todistajatar}}, but this form is ridicuolous to the point of absurdity.  We, like our ancestors who left Sweden and Finland for America, strongly dislike the implication of the newer term \foreignlanguage{finnish}{\textit{todistaja}} --- ``truthteller'' --- one who simply affirms the truth of what the prosecutor presents in court and is not necessarily subject to cross-examination by the defense in the course of trial in the due process of law}}
\newglossaryentry{redress}{name={redress},plural={redresses},description={\foreignlanguage{swedish}{upprättelse}; \foreignlanguage{finnish}{oikaistaminen}; \foreignlanguage{french}{réparation}; a setting right or reparation for a serious matter; implies a rightful compensation for the aggrieved party, not only in terms of money but in terms of restoration of rights, privileges, and status as well as a sharp and heavy rebuke to the opposing party}}
\newglossaryentry{grievance}{name={grievance},description={\foreignlanguage{swedish}{klagomål}; \foreignlanguage{finnish}{valituksen aihe}; \foreignlanguage{french}{doléance}; a serious complaint; the word is built on ``grief'' and implies a weighty matter that must be set right to minimize bloodshed, rebellion, and loss of human life. We have translated the phrase ``a redress of grievances'' to French as \foreignlanguage{french}{\textit{<<~une réparation de doléances,~>>}} fully aware of Civil War demands of African-American former slaves for ``reparations'' and the \foreignlanguage{french}{<<~cahiers de doléances~>>} served on the French monarchy in the final years of the \foreignlanguage{french}{\textit{ancien regime}}. The French word \foreignlanguage{french}{doléance}, too, is built on the word \foreignlanguage{french}{douleur}. We do not feel that we are exaggerating in making such a choice of words}}
\newglossaryentry{held-to-answer}{name={held to answer},description={\foreignlanguage{swedish}{stånden till svars}; \foreignlanguage{finnish}{pidetty vastuussa}; \foreignlanguage{french}{mise à la question}; a deceptively simple phrase in English which seems to correspond to certain idioms or fixed expressions in other languages}}